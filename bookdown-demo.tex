\documentclass[]{book}
\usepackage{lmodern}
\usepackage{amssymb,amsmath}
\usepackage{ifxetex,ifluatex}
\usepackage{fixltx2e} % provides \textsubscript
\ifnum 0\ifxetex 1\fi\ifluatex 1\fi=0 % if pdftex
  \usepackage[T1]{fontenc}
  \usepackage[utf8]{inputenc}
\else % if luatex or xelatex
  \ifxetex
    \usepackage{mathspec}
  \else
    \usepackage{fontspec}
  \fi
  \defaultfontfeatures{Ligatures=TeX,Scale=MatchLowercase}
\fi
% use upquote if available, for straight quotes in verbatim environments
\IfFileExists{upquote.sty}{\usepackage{upquote}}{}
% use microtype if available
\IfFileExists{microtype.sty}{%
\usepackage{microtype}
\UseMicrotypeSet[protrusion]{basicmath} % disable protrusion for tt fonts
}{}
\usepackage{hyperref}
\hypersetup{unicode=true,
            pdftitle={大数据分析及应用团队工作规范},
            pdfauthor={Xin XU},
            pdfborder={0 0 0},
            breaklinks=true}
\urlstyle{same}  % don't use monospace font for urls
\usepackage{natbib}
\bibliographystyle{apalike}
\usepackage{longtable,booktabs}
\usepackage{graphicx,grffile}
\makeatletter
\def\maxwidth{\ifdim\Gin@nat@width>\linewidth\linewidth\else\Gin@nat@width\fi}
\def\maxheight{\ifdim\Gin@nat@height>\textheight\textheight\else\Gin@nat@height\fi}
\makeatother
% Scale images if necessary, so that they will not overflow the page
% margins by default, and it is still possible to overwrite the defaults
% using explicit options in \includegraphics[width, height, ...]{}
\setkeys{Gin}{width=\maxwidth,height=\maxheight,keepaspectratio}
\IfFileExists{parskip.sty}{%
\usepackage{parskip}
}{% else
\setlength{\parindent}{0pt}
\setlength{\parskip}{6pt plus 2pt minus 1pt}
}
\setlength{\emergencystretch}{3em}  % prevent overfull lines
\providecommand{\tightlist}{%
  \setlength{\itemsep}{0pt}\setlength{\parskip}{0pt}}
\setcounter{secnumdepth}{5}
% Redefines (sub)paragraphs to behave more like sections
\ifx\paragraph\undefined\else
\let\oldparagraph\paragraph
\renewcommand{\paragraph}[1]{\oldparagraph{#1}\mbox{}}
\fi
\ifx\subparagraph\undefined\else
\let\oldsubparagraph\subparagraph
\renewcommand{\subparagraph}[1]{\oldsubparagraph{#1}\mbox{}}
\fi

%%% Use protect on footnotes to avoid problems with footnotes in titles
\let\rmarkdownfootnote\footnote%
\def\footnote{\protect\rmarkdownfootnote}

%%% Change title format to be more compact
\usepackage{titling}

% Create subtitle command for use in maketitle
\providecommand{\subtitle}[1]{
  \posttitle{
    \begin{center}\large#1\end{center}
    }
}

\setlength{\droptitle}{-2em}

  \title{大数据分析及应用团队工作规范}
    \pretitle{\vspace{\droptitle}\centering\huge}
  \posttitle{\par}
    \author{Xin XU}
    \preauthor{\centering\large\emph}
  \postauthor{\par}
      \predate{\centering\large\emph}
  \postdate{\par}
    \date{2019-06-18}

\usepackage{booktabs}
\usepackage{amsthm}
\makeatletter
\def\thm@space@setup{%
  \thm@preskip=8pt plus 2pt minus 4pt
  \thm@postskip=\thm@preskip
}
\makeatother

\begin{document}
\maketitle

{
\setcounter{tocdepth}{1}
\tableofcontents
}
\hypertarget{section}{%
\chapter{序}\label{section}}

~~~~~~你好, 欢迎你的加入!

~~~~~~新的数据分析旅程即将开始,预祝你能快速用业务和技术武装自己,一路披荆斩棘,在明动快速成长。\\
~\\
\hspace*{0.333em}\hspace*{0.333em}\hspace*{0.333em}\hspace*{0.333em}\hspace*{0.333em}\hspace*{0.333em}\includegraphics{C:/Users/XuX/Desktop/bookdown-demo-master/cool-car.gif}(←图片源于网络)\\
~\\
\hspace*{0.333em}\hspace*{0.333em}\hspace*{0.333em}\hspace*{0.333em}\hspace*{0.333em}\hspace*{0.333em}文档主要有四部分组成:第一部分是一些基本的入职流程,第二部分是团队各类规范及业务知识,第三部分是需要完成的常规任务,最后是一些其他的说明。\\
~~~~~~现在请自由探索吧, 希望这份文档介绍的内容能帮助你更快的适应工作。:)

\hypertarget{intro}{%
\chapter{新人入职}\label{intro}}

~~~~~~

\begin{enumerate}
\def\labelenumi{\arabic{enumi}.}
\item
  入Q群,并简单介绍自己。
\item
  向行政获取公司邮箱帐号密码。
\item
  向行政获取电脑(如已获取,请忽略此条;若需等待,请先使用私人电脑),安装R、Python3.x及自选IDE(推荐PyCharm)。
\item
  安装Q群数据库相关文件:navicat、robo 3T。
\item
  安装foxmail客户端,添加公司邮箱账户,并将软件通知设置于任务栏可见、开机自动启动。收件服务器pop.exmail.qq.com,发件服务器smtp.exmail.qq.co。
\item
  安装xmind。
\item
  向行政获取公司OA密码,登陆\href{http://192.168.0.212/instance-web/minstone/login}{OA},熟悉工时的填写;事业部Q群里有公司各类规范。
\item
  安装Q群文件中Git、TortoiseGit,在OA发起流程,获取Git账号。
\end{enumerate}

\hypertarget{section-1}{%
\chapter{团队规范及业务知识}\label{section-1}}

\begin{enumerate}
\def\labelenumi{\arabic{enumi}.}
\item
  要具备的基本技能:Python 或 R,简单的SQL语句。
\item
  需要知道的\href{http://git.minstone.com.cn/dataanalysisteam/minstone_data_analyze_team/work_standard}{工作规范},包括代码规范、制图规范、文档记录规范、代码版本管理规范等。
\item
  需要了解熟悉的\href{}{业务知识}:行政执法、环保(Q群文件-环保业务)、政务\ldots{}\ldots{}
\item
  一些有用的\href{http://git.minstone.com.cn/dataanalysisteam/minstone_data_analyze_team/QuickStart}{tips}:数据库连接、R包技巧、Git常用操作\ldots{}\ldots{}
\end{enumerate}

\hypertarget{section-2}{%
\chapter{常规任务}\label{section-2}}

\begin{enumerate}
\def\labelenumi{\arabic{enumi}.}
\item
  每天上班,请梳理今日工作计划,合理安排自己的时间。(团队暂时使用Teambition。1.注册;2.新建项目,并将经理加入项目;3.以日期命名任务,上班时新增,下班时回顾任务。)
\item
  每天下班离开前,登陆公司\href{http://192.168.0.212/instance-web/minstone/login}{OA},填写当日工作工时。合同项目需选择相应项目填写工时。
\item
  每天下班离开前,向git提交代码。
\item
  每周一中午12点前,完成并以邮件形式发出上周\href{http://git.minstone.com.cn/dataanalysisteam/minstone_data_analyze_team/work_standard/blob/master/WeeklyGuide_V1.0.md}{工作周报}(模版请询问)。标注\%,以说明本周工作任务完成情况。
\end{enumerate}

\hypertarget{section-3}{%
\chapter{培训}\label{section-3}}

\\
\textbf{目标:}\\
\hspace*{0.333em}\hspace*{0.333em}\hspace*{0.333em}\hspace*{0.333em}\hspace*{0.333em}\hspace*{0.333em}知识分享,共同进步。\\
\textbf{时间:}\\
\hspace*{0.333em}\hspace*{0.333em}\hspace*{0.333em}\hspace*{0.333em}\hspace*{0.333em}\hspace*{0.333em}每两周一次(暂时)\\
\textbf{讲师:}\\
\hspace*{0.333em}\hspace*{0.333em}\hspace*{0.333em}\hspace*{0.333em}\hspace*{0.333em}\hspace*{0.333em}团队成员,依次轮换。\\
\textbf{内容: }\\
\hspace*{0.333em}\hspace*{0.333em}\hspace*{0.333em}\hspace*{0.333em}\hspace*{0.333em}\hspace*{0.333em}好用有用的包介绍、代码技巧、可复用函数等。\\
\textbf{地点:}\\
\hspace*{0.333em}\hspace*{0.333em}\hspace*{0.333em}\hspace*{0.333em}\hspace*{0.333em}\hspace*{0.333em}当次培训讲师需至少提前一天在OA预定会议室,并在团队Q群发布培训通知。\\
~~~~~~通知模板:\\
培训:内容写在这里,请明确培训内容\\
时间:时间写在这里,几月几日(周几) 00:00\\
地点:会议室x\\
参会人员需提前准备:需提前准备的内容或预习的材料写在这里,如有材料需附以文档。

\hypertarget{section-4}{%
\chapter{其他}\label{section-4}}

\begin{enumerate}
\def\labelenumi{\arabic{enumi}.}
\item
  请假:请假请提前告知经理,请假天数达3天及以上最好提前5天申请,以便安排工作。获口头得批准后,在OA走流程申请《广州请假申请表》。请假时请告知助理备案。特殊情况可特殊处理。
\item
  当发送材料或沟通重要事务,对象是公司内其他事业部及部门的同事时,需以邮件形式,并抄送经理;当部门内交付工作结果或发送其他重要文件时,需以邮件形式;当发送材料对象是公司外部客户及人员时,需获得经理同意,并以邮件形式发送,抄送经理。
\item
  加班:
\end{enumerate}

\begin{verbatim}
if 晚上加班 & 下班时间 > 8pm:
    if 想点公司的订餐:
        要在界面把定位改到a栋再点。
        每天17: 00截止。
    else:
        出去吃报销。
else:
    拜拜,明天见~~
\end{verbatim}

\bibliography{book.bib,packages.bib}


\end{document}
